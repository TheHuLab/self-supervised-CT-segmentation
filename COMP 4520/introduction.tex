\section{INTRODUCTION}

\IEEEPARstart{C}{OVID-19}  is a newly identified disease that is very contagious and has been rapidly spreading across different countries around the world. The virus that was first identified in Wuhan has now infected more than 3.5 million people around the whole world and causes more than 245,000 deaths. Common symptoms from COVID-19 are fever, dry cough, but in more serious cases, patients can experience difficulty in breathing. As more people are infected, communities that have been in close contact with infected patients are getting tested for COVID-19. The test used to carry out the test for COVID-19 uses PCR(Polymerase Chain Reaction) test which could take several days for the test results to be available as the test samples are sent to a centralized lab for analysis and can be time consuming. There is a limited number of supplies of PCR tests which is a bottleneck for testing to be efficient. Several alternative methods have been considered to test patients that are COVID-19 positive including CT scan of the lungs. CT scans of the lungs are faster and easier to detect COVID-19 presence in patients. As the number of infected patients increases exponentially, it can be hard to provide testing scans for patients because of the limited number of doctors. It is recommended that Artificial Intelligence systems are used to analyse the CT scans of lung patients to determine the severity of COVID-19 and monitor the disease progression as well as to compensate for the high number of patients. Specifically, we propose using deep learning to analyze and create a pixel-level segmentation of CT scan images of patients’ lungs to determine the severity of COVID-19 in their lungs. In order to obtain the severity score, the model is first trained to segment the infected region of the lungs. Then, t	he severity score will be calculated by calculating the overlapping ratio between the segmented region for infected regions and the parenchyma of the lungs. 
