\section{INTRODUCTION}

\IEEEPARstart{C}{OVID-19} is a newly identified disease that is very contagious and has been rapidly spreading across different countries around the world. The virus that was first identified in Wuhan has now infected more than 3.5 million people around the whole world and causes more than 245,000 deaths. Common symptoms from COVID-19 are fever, dry cough, but in more serious cases, patients can experience difficulty in breathing. As more people are infected, communities that have been in close contact with infected patients are getting tested for COVID-19. The test used to carry out the test for COVID-19 uses Polymerase Chain Reaction (PCR) test which could take several days for the test results to be available as the test samples are sent to a centralized lab for analysis and can be time-consuming. There is a limited number of supplies of PCR tests which is a bottleneck for testing to be efficient. Several alternative methods have been considered to test patients that are COVID-19 positive including a computerized tomography (CT) scan of the lungs. CT scans of the lungs are faster and easier to detect COVID-19 presence in patients. As the number of infected patients increases exponentially, it can be hard to provide testing scans for patients because of the limited number of doctors. It is recommended that Artificial Intelligence systems are used to analyze the CT scans of lung patients to determine the infected region of the lungs with COVID-19 and monitor the disease progression as well as to compensate for the high number of patients. Convolutional Neural Network (CNN) \cite{ref1} is an important technique to be used in image processing as CNN is able to automatically captures useful features instead of handcrafting features to be used to evaluate on the segmentation of the CT lung images. Fan et al. \cite{ref2} developed InfNet that uses CNN and fully supervised method to predict the segmentation of ground-glass opacities and consolidation. Fan et al. also incorporated semi-supervised learning to enlarge the limited number of training samples for CT lung image segmentation. However, deep learning requires a large of number of samples to be able to achieve good performance. Fan et al. uses pseudo labelling as semi-supervised learning to migitate the limited number of samples but pseudo labelling can take up to 3 times the time taken to train the network when undergoing fully supervised learning.  Therefore, we propose using self-supervised deep learning to analyze and create a pixel-level segmentation of CT scan images of patients’ lungs to determine the infected area of the CT lung images that includes ground-glass opacities and consolidation. Our self-supervised learning method integrated into InfNet is less time consuming than pseudo labelling as pseudo labelling requires training InfNet on the labeled data, evaluating InfNet on the unlabeled data, then re-trained InfNet on the total dataset. Our \textit{key contribution} in this paper is to integrate self-supervision into an existing network to improve the performance of the original network. We extend the work of InfNet as InfNet is one of the high performing model that includes CNN and several techniques to segment the ground-glass opacities and the consolidation area of the CT lung images. We show that integration self-supervised learning to InfNet improves the performance of InfNet.